%\iffalse
\let\negmedspace\undefined
\let\negthickspace\undefined
\documentclass[journal,12pt,twocolumn]{IEEEtran}
\usepackage{cite}
\usepackage{amsmath,amssymb,amsfonts,amsthm}
\usepackage{algorithmic}
\usepackage{graphicx}
\usepackage{textcomp}
\usepackage{xcolor}
\usepackage{txfonts}
\usepackage{listings}
\usepackage{enumitem}
\usepackage{mathtools}
\usepackage{gensymb}
\usepackage{comment}
\usepackage{caption}
\usepackage[breaklinks=true]{hyperref}
\usepackage{tkz-euclide} 
\usepackage{listings}
\usepackage{gvv}                                        
\def\inputGnumericTable{}                                 
\usepackage[latin1]{inputenc}                                
\usepackage{color}                                            
\usepackage{array}                                            
\usepackage{longtable}                                       
\usepackage{calc}                                             
\usepackage{multirow}                                         
\usepackage{hhline}                                           
\usepackage{ifthen}                                           
\usepackage{lscape}

\newtheorem{theorem}{Theorem}[section]
\newtheorem{problem}{Problem}
\newtheorem{proposition}{Proposition}[section]
\newtheorem{lemma}{Lemma}[section]
\newtheorem{corollary}[theorem]{Corollary}
\newtheorem{example}{Example}[section]
\newtheorem{definition}[problem]{Definition}
\newcommand{\BEQA}{\begin{eqnarray}}
\newcommand{\EEQA}{\end{eqnarray}}
\newcommand{\define}{\stackrel{\triangle}{=}}
\theoremstyle{remark}
\newtheorem{rem}{Remark}
\begin{document}

\bibliographystyle{IEEEtran}
\vspace{3cm}

\title{GATE 2021 ee 43}
\author{EE23BTECH11022 - G DILIP REDDY}
\maketitle

\bigskip

\renewcommand{\thefigure}{\arabic{figure}}
\renewcommand{\thetable}{\arabic{table}}
\textbf{Question}:\\
Consider a continuous-time signal $x\brak{t}$ \,defined by $x\brak{t}=0$\,for $\abs{t}>1$, and $x\brak{t}=1-\abs{t}$ for $\abs{t}\le 1$. Let the Fourier transform of $x\brak{t}$ be defined as $X\brak{\omega}=\int_{-\infty}^{\infty}x\brak{t}e^{-j\omega t} dt$. The maximum magnitude of $X\brak{\omega}$ is $\hbox to 4em{\thinspace\hrulefill\thinspace}$.
\hfill{(GATE 2021 EE 43)}
\\\\
\solution
%\fi
\begin{align}
X\brak{\omega}&=\int_{-\infty}^{\infty}x\brak{t}e^{-j\omega t} dt\\
X\brak{\omega}&=\int_{-1}^{1}\brak{1-\abs{t}}e^{-j\omega t} dt\\
X\brak{\omega}&=\int_{-1}^{1}e^{-j\omega t} 
dt - \int_{-1}^{1}\abs{t}e^{-j\omega t} dt\\
X\brak{\omega}&=2\int_{0}^{1}\cos\brak{{\omega t}}
dt - 2\int_{0}^{1}t\cos\brak{\omega t} dt\\
X\brak{\omega}&=2\frac{\sin\brak{{\omega }}}{\omega}
- 2\sbrak{\frac{\sin\brak{\omega }}{\omega}+\frac{\cos\brak{\omega }}{\omega^2}-\frac{1}{\omega ^2}}\\
X\brak{\omega}&=2\frac{1-\cos\brak{\omega }}{\omega^2}\\
X\brak{\omega}&=2\frac{2\sin^2\brak{\frac{\omega}{2}}}{\omega^2}\\
\omega & \to 0 \implies X\brak{\omega}\to1
\end{align}
Maximum of magnitude of $X\brak{\omega}=1$
\end{document}